%%%%%%%%%%%%%%%%%%%%%%%%%%%%%%%%%%%%%%%%%%%%%%%%%%%%%%%%%%%%%%%%%%%%%%%%%%%%%%%%
% Medium Length Graduate Curriculum Vitae
% LaTeX Template
% Version 1.2 (3/28/15)
%
% This template has been downloaded from:
% http://www.LaTeXTemplates.com
%
% Original author:
% Rensselaer Polytechnic Institute 
% (http://www.rpi.edu/dept/arc/training/latex/resumes/)
%
% Modified by:
% Daniel L Marks <xleafr@gmail.com> 3/28/2015
%
% Important note:
% This template requires the res.cls file to be in the same directory as the
% .tex file. The res.cls file provides the resume style used for structuring the
% document.
%
%%%%%%%%%%%%%%%%%%%%%%%%%%%%%%%%%%%%%%%%%%%%%%%%%%%%%%%%%%%%%%%%%%%%%%%%%%%%%%%%

%-------------------------------------------------------------------------------
%	PACKAGES AND OTHER DOCUMENT CONFIGURATIONS
%-------------------------------------------------------------------------------

%%%%%%%%%%%%%%%%%%%%%%%%%%%%%%%%%%%%%%%%%%%%%%%%%%%%%%%%%%%%%%%%%%%%%%%%%%%%%%%%
% You can have multiple style options the legal options ones are:
%
%   centered:	the name and address are centered at the top of the page 
%				(default)
%
%   line:		the name is the left with a horizontal line then the address to
%				the right
%
%   overlapped:	the section titles overlap the body text (default)
%
%   margin:		the section titles are to the left of the body text
%		
%   11pt:		use 11 point fonts instead of 10 point fonts
%
%   12pt:		use 12 point fonts instead of 10 point fonts
%
%%%%%%%%%%%%%%%%%%%%%%%%%%%%%%%%%%%%%%%%%%%%%%%%%%%%%%%%%%%%%%%%%%%%%%%%%%%%%%%%

\documentclass[margin]{res}  

% Default font is the helvetica postscript font
\usepackage{helvet}
\usepackage{hyperref}
\hypersetup{
  colorlinks, linkcolor=red, urlcolor=blue
}
% Increase text height
\textheight=700pt

\begin{document}

%-------------------------------------------------------------------------------
%	NAME AND ADDRESS SECTION
%-------------------------------------------------------------------------------

\name{\LARGE Raghunandana J Reddy} 
% Note that addresses can be used for other contact information:
% -phone numbers
% -email addresses
% -linked-in profile

\address{rjayaramared@cs.stonybrook.edu\\+1 (631) 590 9685 \\ \url{https://www.linkedin.com/in/raghunanj/ }\\}
\address{700 Health Sciences Dr \#927 \\ Stony Brook, NY 11790 \\ \url{https://github.com/raghunanj/ }}

% Uncomment to add a third address
%\address{Address 3 line 1\\Address 3 line 2\\Address 3 line 3}
%-------------------------------------------------------------------------------

\begin{resume}

%-------------------------------------------------------------------------------
%	EDUCATION SECTION
%-------------------------------------------------------------------------------
\section{EDUCATION}
\textbf{Stony Brook University}, NY  \hfill August 2017 - June 2019 \\
{\sl MS in Computer Science}, Aug 2017 - June 2019 
\begin{itemize}
\item Coursework: Operating Systems, Computer Vision, Network Security, Analysis of Algorithms
\end{itemize}
\textbf{PES Institute of Technology}, Bengaluru, INDIA \hfill September 2011 - June 2015 \\
{\sl BE in Computer Science},  \hfill GPA: 8.93/10.0 
\begin{itemize}
\item Coursework: Analysis and Design of Algorithms, Computer Networks, Unix Programming, Fuzzy logic, Compiler Design, Principles of Programming Languages. 
\end{itemize}
%-------------------------------------------------------------------------------
 
%-------------------------------------------------------------------------------
%	PROJECTS SECTION
%-------------------------------------------------------------------------------
%-------------------------------------------------------------------------------


%-------------------------------------------------------------------------------
%	COMPUTER SKILLS SECTION
%-------------------------------------------------------------------------------
\section{TECHNOLOGY \\ SKILLS} 

\ C, Python, Java, R, NumPy, SciPy, Scikit-Learn, OpenCV, C++(moderate). \\
\ IntelliJ, PyCharm, Git, MATLAB, Eclipse.\\
%-------------------------------------------------------------------------------

%-------------------------------------------------------------------------------
%	EXPERIENCE SECTION
%-------------------------------------------------------------------------------
% Modify the format of each position
\begin{format}
\title{l}\employer{r}\\
\dates{l}\location{r}\\
\body\\
\end{format}
%-------------------------------------------------------------------------------
\section{EXPERIENCE}

\textbf{Member of Technical Staff-Tech Analyst 2} \hfill July 2015 - June 2017 \\
Juniper Networks Inc, Bengaluru, INDIA
\begin{itemize}
\item Application development in IGNITION 14.7, a PaaS for data Management.
\item Developed the designed enhancements in VESPER, a data cleansing initiative, 
\item Technologies Used: Java, Python, SAP ABAP, JavaScript, VBScript. 
\end{itemize} 

\textbf{MTS Intern} \hfill January 2015 - June 2015 \\
Juniper Networks Inc, Bengaluru, INDIA 
\begin{itemize} \itemsep -2pt % Reduce space between items
\item Designed and developed the automation scenarios in VBScript and Python.
\item Ensured that the performance and optimality of the designed solution are meeting a logical end by building a simulator using robot framework in Python.
\end{itemize}
 
\textbf{Research Intern} \hfill May 2014 - August 2014\\
IIT Guwahati, INDIA
\begin{itemize} 
\item Indian Academy of Sciences' fellow advised by Dr. Partha Sarathi Mandal.
\item Worked on Coverage Hole minimization in wireless sensor networks(WSNs) and developed a simulator for the algorithm we developed using frameworks in Python.
\end{itemize} 


\section{ACADEMIC PROJECTS}
\textbf{Merging EventShop and AsterixDB}  \hfill Advised by Dr.Dinkar Sitaram
\begin{itemize} 
\item Fused EventShop - an OpenSource query processing application and AsterixDB, an OpenSource Big Data Management System.
\item Ensured and demonstrated that the scale, diversity and complexity of data generated by EventShop can be handled optimally by AsterixDB compared to a traditional RDBMS.
\end{itemize}

\textbf{Big Data Parallelization using GPU}  \hfill Advised by Prof. Ashok Raman
\begin{itemize} 
\item Parallelized the mahout K-Means clustering algorithm to run it on a GPU system and hence improve the performance.
\item Used aparapi library to convert java bytecode to opencl and wrote our own library in Jcuda to invoke the GPU kernel code.
\end{itemize}


\section{AWARDS AND HONORS}
\begin{itemize}
\item Received the MHRD merit scholarship by the Ministry of HRD, India during 2011-2015.
\item Distinction award in the years 2013,2014,2015 at PES Institute of Technology. 
\item Incredible Talent Team award in 2016-2017 at Juniper Networks Inc India.
\end{itemize}


%-------------------------------------------------------------------------------
\end{resume}
\end{document}